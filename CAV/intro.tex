\section{Introduction}

Acme++ is a tool for deciding properties of some algebraic structures called stabilization monoids,
with two motivations in mind:
\begin{itemize}
\item provide an effective tool to solve the Star-height problem
\item provide a tool to solve the value 1 problem for probabilistic automata.
\end{itemize}

\textbf{The Star-height problem}

The Star-height problem is the following: given a regular language $L$ and an integer $k$, is there a regular expression for $L$ with at most $k$ nesting of the Kleene star?
An excellent introduction about the Star-height problem is given in \cite{Kirsten05},
which mentions some of the important industrial applications as
speech recognition \cite{Mohri97}, database theory \cite{GT01}, and image compression \cite{CK93,KMT04}.
This problem was considered as one of the most difficult problems in the theory of recognizable languages
and it took 25 years before being solved by Hashiguchi \cite{Hashiguchi88}.
Implementing Hashiguchi algorithm is hopeless: the algorithm proceeds by enumeration of certain expressions
and has a terrible complexity \cite{LS02}.
It took another 22 years before an algorithm with a better algorithmic complexity was given by Kirsten in \cite{Kirsten05}.
This approach using weighted automata was generalized and inspired the rich theory of regular cost functions \cite{Colcombet09}, which allows to solve other problems reducible to boundedness of automata with counters \cite{CL08sh,CL08b,CKLB13}.

Acme++ aims at solving the Star-height problem for practical applications,
albeit the doubly exponential space complexity (simply exponential for fixed $k$) of Kirsten's algorithm is a challenge to tackle.

We use an optimization from \cite{CL08sh}: the structure of Subset Automata, whose algebraic properties allow minimization.
\smallskip

\textbf{The value 1 problem}

Probabilistic automata are a versatile tool widely used in speech recognition as well as a modelling tool 
for the control of systems with partial observation or no observation.
Probabilistic automata are a natural extension on automata on finite words,
with probabilistic transitions, see \cite{Rabin63} for an introduction.
The value 1 problem is natural when probabilistic automata are used as models
of systems controlled by a blind controller, who is in charge of choosing the
sequence of input letters in order to maximize the acceptance
probability, see \cite{FGO12} for an introduction.
\smallskip

\textbf{Stabilization monoids}

Stabilization monoids are the key mathematical object needed to solve the two questions we are interested in.
The Star-height problem is related to the limitedness problem of nested distance automata.
A seminal paper by Simon \cite{Sim94} unifies several results using monoids of matrices, providing in particular a combinatorial tool called the forest factorization theorem. 
Then Kirsten showed how monoids could be used to solve the Star-height problem using nested distance desert automata,
giving an alternate proof to the very intricate proof of Hashigushi.
Desert automata are automata with ranked counters that can be either incremented and reset,
with the extra condition that if a counter is reset all counters of lower rank are reset as well.
Colcombet extended further the work of Kirsten introducing stabilization monoids and generalizing regular language theory into cost function theory, including algebraic and logical characterizations \cite{Colcombet09,CKL10,Kup14}. The theory of regular cost functions can also be used to solve the Star-height problem on finite trees \cite{CL08sh}, as well as others problems related to regular languages.
The techniques of Simon were adapted to solve the value 1 problem for probabilistic automata \cite{FGO12}.


Acme++ is a set of software tools to handle stabilization monoids, they can generate a stabilization monoid from its generators and test properties of these monoids, in particular the limitedness and the existance of value 1 witnesses.

Our results:
TODO : provide a description of experimental results.