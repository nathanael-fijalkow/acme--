\section{Stabilization Monoids for $B$- and Probabilistic Automata}

The notion of stabilization monoids appears in two distinct contexts.
It has first been developed in the theory of regular cost functions,
introduced by Colcombet~\cite{Colcombet09,Colcombet13}.
The underlying ideas have then been transferred to the setting of probabilistic automata~\cite{FGO12}.

\subsection{Stabilization Monoids in the Theory of Regular Cost Functions}

At the heart of the theory of regular cost functions lies the equivalence between different formalisms:
a logical formalism, cost MSO, two automata model, $B$- and $S$-automata, and an algebraic counterpart, stabilization monoids.

Here we briefly describe the model of $B$-automata, and their transformations to stabilization monoid.
This automaton model generalizes the non-deterministic automata by adding a finite set of counters.
Instead of accepting or rejecting a word as a non-deterministic automaton does, 
a $B$-automaton associates an integer value to each input word.
Formally, a $B$-automaton is a tuple $\A = \perm{A,Q,\Gamma,I,F,\Delta}$, where $A$ is a finite alphabet, $Q$ is a finite set of states, 
$\Gamma$ is a finite set of counters, $I \subseteq Q$ is the set of initial states, $F \subseteq Q$ is the set of final states, 
and $\Delta \subseteq Q \times A \times \set{\ic,\e,\re}^\Gamma \times Q$ is the set of transitions.

A transition $(p,a,\tau,q)$ allows the automaton to go from state $p$ to state $q$ while reading letter $a$ and performing action $\tau(\gamma)$ on counter $\gamma$. 
Action $\ic$ increments the current counter value by $1$, $\e$ leaves the counter unchanged, and $\re$ resets the counter to $0$.

%A run of the automaton $\A$ on a word $w = a_1 a_2 \dots a_n$ is a sequence $\rho = q_0,\tau_1,q_1,\tau_2,\dots, \tau_n, q_n$ such that $q_0 \in I, q_n \in F$, 
%and for all $i \in [1,n]$, $(q_{i-1},a_i,\tau_i,q_i) \in \Delta$.
The value of a run is the maximal value assumed by any of the counters during the run.
The semantics of a $B$-automaton $\A$ is defined on a word $w$ by 
$\sem{\A}(w) = \inf\set{\val{\rho} \mid \rho\text{ is a run of } \A \text{ on } w}$.
In other words, the automaton uses the non determinism to minimize the value among all runs.
In particular, if $\A$ has no run on $w$, then $\sem{\A}(w) = \infty$.

%If $\A$ is a $B$-automaton, its semantics $\sem{\A}$ is usually viewed as a cost function, \textit{i.e.} exact values are ignored, 
%and just boundedness properties are considered. 
%Consequently, we say that two $B$-automata $\A_1$ and $\A_2$ are equivalent if $\sem{\A_1}$ and $\sem{\A_2}$ are bounded on the same sets of words.

The main decision problem in the theory of regular cost functions is the limitedness problem.
We say that a $B$-automaton $\A$ is \emph{limited} if there exists $N$ such that for all words $w$, if $\sem{\A}(w) < \infty$, then $\sem{\A}(w) < N$.

\vskip1em
One way to solve the limitedness problem is by computing the stabilization monoid.
It is a monoid of matrices over the semiring of sets of counter actions $\set{\ic,\e,\re,\omega}^\Gamma$.
There are two operations on matrices: a binary composition called product, giving the monoid structure,
and a unary operation called stabilization.
The stabilization monoid of a $B$-automaton is the set of matrices containing the matrices corresponding to each letter,
and closed under the two operations, product and stabilization.
As shown in~\cite{Colcombet09,Colcombet13}, the stabilization monoid of a $B$-automaton $\A$ contains an unlimited witness
if and only if it is not limited,
implying a conceptually simple solution to the limitedness problem: compute the stabilization monoid
and check for the existence of unlimited witnesses.

%Following the celebrated success for the star-height problem, whose decidability has been established by
%reducing it to the limitedness problem of $B$-automata~\cite{Hashiguchi88,Kirsten05}, 
%several problems have been reduced to this problem.
%For instance, we implemented the solution of the finite power property:
%given a regular language $L \subseteq A^*$, does there exist $n \in \N$
%such that $L^* = L^ 0 + L^1 + \cdots + L^n$?
%We reduce this problem to a limitedness problem for $B$-automata, following \cite{Kirsten02}.

\subsection{Stabilization Monoids for Probabilistic Automata}

The notion of stabilization monoids also appeared for probabilistic automata, for the Markov Monoid Algorithm.
This algorithm was introduced in~\cite{FGO12} to partially solves the value $1$ problem: given a probabilistic automaton $\A$,
does there exist $(u_n)_{n \in \N}$ a sequence of words such that $\lim_n \mathbb{P}_\A(u_n) = 1$?

Although the value $1$ problem is undecidable, it has been shown that
the Markov Monoid Algorithm correctly determines whether a probabilistic automaton has value $1$
under the \textit{leaktight} restriction.
It has been recently shown that all classes of probabilistic automata for which the value $1$ problem has been shown decidable 
are included in the class of leaktight automata~\cite{FGKO14},
hence the Markov Monoid Algorithm is the \textit{most correct} algorithm known to (partially) solve the value $1$ problem.

As for the case of $B$-automata, the stabilization monoid of a probabilistic automaton
is the set of matrices containing the matrices corresponding to each letter,
and closed under the two operations, product and stabilization.

Note that the main point is that both the products and the stabilizations depend on which type of automata is considered,
$B$-automata or probabilistic automata.