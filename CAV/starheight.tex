\newcommand{\cB}{\mathcal B}
\newcommand{\trans}[1]{\stackrel{#1}{\longrightarrow}}

\section{The Star-Height algorithm}

The latest algorithm in the literature for computing star-height is designed for tree automata \cite{CL08sh}, but we will use it here in the special case of words. The main improvement over the previous algorithm from \cite{Kirsten05} is the identification of the structure of Subset Automata, which allows minimization.

We describe here briefly the ideas of the algorithm.

\subsection{Subset automata}

\begin{definition}\cite{CL08sh}
A subset automaton $\A$ is a deterministic automaton with additional $\epsilon$-transition, such that $\epsilon$-transition encode a lattice structure on the set space of $\A$. More precisely, $\epsilon$-transitions define a partial order on states, and for all states $p,q$, there are states $x,y$ such that $x\trans{\epsilon}p, x\trans{\epsilon}q, p\trans{\epsilon}y$ and $q\trans{\epsilon}y$.
\end{definition}

Due to the algebraic nature of their definition, subset automata can be minimized in a similar way to deterministic automata.

\begin{theorem}\cite{CL08sh}
Any language can be recognized by a subset automaton, obtained by a powerset construction from a non-deterministic automaton for the complement language.
\end{theorem}

\subsection{Reduction to limitedness}

Let $\A=\perm{A,Q,q_0,F_\A,\Delta_\A}$ be a subset automaton for a language $L$, and $k\in \mathbb N$. We recall that there are $\epsilon$-transitions, so $\Delta_\A\subseteq Q\times (A\cup\{epsilon\})\times Q$.

We build a $B$-automaton $\cB$ with $k+1$ counters $\gamma_0,\gamma_1,\dots,\gamma_k$, and states $Q_\cB=\bigcup_{i=1}^{k+1} Q^i$ that we view as a subset of $Q^*$.

Let $j\in[0,k]$, we will note $R_j$ the counter action performing $\re$ on counters $\gamma_p$ with $p\geq j$ and $\e$ on counters $\gamma_p$ with $p<j$. Simlarly, we will note $I_j$ the counter action performing $\re$ on counters $\gamma_p$ with $p> j$, $\i$ on counter $\gamma_j$, and $\e$ on counters $\gamma_p$ with $p<j$. We finally note $\e$ the action performing $\e$ on all counters.
These particular counter actions are called \emph{hierarchical}, they correspond to the nested distance automata from \cite{Kirsten05}. Their advantage over general actions on multiple counters is that there is a total order or preference, namely $R_0\leq R_1\leq\dots R_k\leq \e\leq I_0\leq\ I_1\leq\dots\leq I_k$. 

The automaton $\cB$ is defined as follows:

\begin{itemize}
\item The initial state is $q_0$ (word of length $1$).
\item A state $wp$ is final if and only if $p\in F_\A$.
\item If $(p,a,q)\in\Delta$ and $w\in Q^{\leq k}$, there is a transition $wp\trans{a:I_{|w|}}wq$ in $\cB$. If $a=\epsilon$, the action may equivalently be equivalently replaced by $\e$, as we do in the implementation.
\item If $w\in Q^{\leq k-1}$ and $p\in Q_\A$, there is a transition $wp\trans{\epsilon:R_{|wp|}}wpp$ in $\cB$.
\item If $w\in Q^{\leq k-1}$ and $q,p\in Q_\A$, there is a transition $wpq\trans{\epsilon:R_{|wp|}}wq$ in $\cB$.
\end{itemize}


The following theorem guarantees the correctness of the reduction.
\begin{theorem}\cite{CL08sh}
The automaton $\cB$ is limited if and only if $L$ is expressible with a regular expression of star-height $k$.
\end{theorem}

Therefore, for any fixed $k$, we can decide in EXPSPACE whether a regular language has star-height $k$, if its is given via a nondeterministic automaton for the complement (or via a deterministic automaton).

The algorithm is as follow: build a subset automaton $\mathcal A$ via a powerset construction (exponential space), minimize it, then build the automaton $\cB$ as above, and finally test for limitedness of $\cB$ by building its stabilization monoid and looking for an unlimitedness witness.
